\documentclass[11pt]{article}

\usepackage{amsfonts}
\usepackage{amssymb}
\usepackage{amsthm}
\usepackage{fancybox}
\usepackage{amsmath}
\usepackage{fullpage}
\usepackage{setspace}
\usepackage{mathtools}
\usepackage{times}
\usepackage{color}
\usepackage{hyperref}
\usepackage{polynom}
\usepackage[utf8]{inputenc}

\usepackage[pdftex]{graphicx}
\usepackage{multicol}
\usepackage{fancyhdr}
\usepackage{titlesec}

\setlength{\fboxsep}{1em}
\setlength{\parindent}{0pt}
\setlength{\headheight}{14pt}
\renewcommand{\headrulewidth}{1pt}
\renewcommand{\footrulewidth}{1pt}
\setlength{\headsep}{14pt}

\begin{document}

\pagestyle{fancy}
\fancyhead{}
\fancyhead[L]{CSCC43}
\fancyhead[R]{PROJECT: MyBnB}
\fancyfoot{}
\fancyfoot[R]{\thepage}
\fancyfoot[L]{Pasa Ali Aslan}

\section*{Project Description}

MyBnB is a home sharing service platform like AirBnb in which users can list or book
residences around the world. Recognizing its mass audience, MyBnB is planned to be
a full-stack application with web and mobile support.

Considering course objectives and limited time-frame, this project (its source code)
aims to provide the HTTP server (Java) and database (MySQL) portion of MyBnB with minimal functionality given by the
project requirements. Similarly, Java SQL adapters are used with raw MySQL queries
to enable the developer to strenthen his understanding on the database transactions. No frameworks or ORMs are used.
Nevertheless, the server code is written as systematically as possible to keep a consistent high code
quality.


\section*{Source Code and Setup}

The source code is provided in [URL GOES HERE]. 'README.md' provides all the necessary
information (or references the documents with the specific information), including the setup
process on a local machine, server API documentation, and blueprints of the database system
(ER diagrams, schemas, assumptions, etc.).

After one completes the setup process, he can run HTTP queries on terminal or (more simply)
on Postman. Each endpoint is well documented under ./doc/apis

\section*{Assumptions}

\section*{ER Diagram}

\section*{Schemas and Constraints}

\section*{Limitations and Areas of Improvement}




\end{document}
